%%%%%%%%%%%%%%%%%
% This is an sample CV template created using altacv.cls
% (v1.3, 10 May 2020) written by LianTze Lim (liantze@gmail.com). Now compiles with pdfLaTeX, XeLaTeX and LuaLaTeX.
% This fork/modified version has been made by Nicolás Omar González Passerino (nicolas.passerino@gmail.com, 15 Oct 2020)
%
%% It may be distributed and/or modified under the
%% conditions of the LaTeX Project Public License, either version 1.3
%% of this license or (at your option) any later version.
%% The latest version of this license is in
%%    http://www.latex-project.org/lppl.txt
%% and version 1.3 or later is part of all distributions of LaTeX
%% version 2003/12/01 or later.
%%%%%%%%%%%%%%%%

%% If you need to pass whatever options to xcolor
\PassOptionsToPackage{dvipsnames}{xcolor}

%% If you are using \orcid or academicons
%% icons, make sure you have the academicons
%% option here, and compile with XeLaTeX
%% or LuaLaTeX.
% \documentclass[10pt,a4paper,academicons]{altacv}

%% Use the "normalphoto" option if you want a normal photo instead of cropped to a circle
% \documentclass[10pt,a4paper,normalphoto]{altacv}

%% Fork: CV dark mode toggle enabler to use a inverted color palette.
%% Use the "darkmode" option if you want a color palette used to 
% \documentclass[10pt,a4paper,darkmode]{altacv}

\documentclass[10pt,a4paper,ragged2e,withhyper]{altacv}

%% AltaCV uses the fontawesome5 and academicons fonts
%% and packages.
%% See http://texdoc.net/pkg/fontawesome5 and http://texdoc.net/pkg/academicons for full list of symbols. You MUST compile with XeLaTeX or LuaLaTeX if you want to use academicons.

% Change the page layout if you need to
\geometry{left=1.2cm,right=1.2cm,top=1cm,bottom=1cm,columnsep=0.75cm}

% The paracol package lets you typeset columns of text in parallel
\usepackage{paracol}

% Change the font if you want to, depending on whether
% you're using pdflatex or xelatex/lualatex
\ifxetexorluatex
  % If using xelatex or lualatex:
  \setmainfont{Roboto Slab}
  \setsansfont{Lato}
  \renewcommand{\familydefault}{\sfdefault}
\else
  % If using pdflatex:
  \usepackage[rm]{roboto}
  \usepackage[defaultsans]{lato}
  % \usepackage{sourcesanspro}
  \renewcommand{\familydefault}{\sfdefault}
\fi

% Fork: Change the color codes to test your personal variant on any mode
\ifdarkmode%
  \definecolor{PrimaryColor}{HTML}{0F52D9}
  \definecolor{SecondaryColor}{HTML}{3F7FFF}
  \definecolor{ThirdColor}{HTML}{F3890B}
  \definecolor{BodyColor}{HTML}{ABABAB}
  \definecolor{EmphasisColor}{HTML}{ABA2A2}
  \definecolor{BackgroundColor}{HTML}{242424}
  \definecolor{AboutColor}{HTML}{ABABAB}
\else%
  \definecolor{PrimaryColor}{HTML}{001F5A}
  \definecolor{SecondaryColor}{HTML}{0039AC}
  \definecolor{ThirdColor}{HTML}{F3890B}
  \definecolor{BodyColor}{HTML}{666666}
  \definecolor{EmphasisColor}{HTML}{2E2E2E}
  \definecolor{BackgroundColor}{HTML}{E2E2E2}
  \definecolor{AboutColor}{HTML}{222222}
\fi%

\colorlet{name}{PrimaryColor}
\colorlet{tagline}{PrimaryColor}
\colorlet{about}{AboutColor}
\colorlet{heading}{PrimaryColor}
\colorlet{headingrule}{ThirdColor}
\colorlet{subheading}{SecondaryColor}
\colorlet{accent}{SecondaryColor}
\colorlet{emphasis}{EmphasisColor}
\colorlet{body}{BodyColor}
\colorlet{hyperlink}{PrimaryColor}
\pagecolor{BackgroundColor}

% Change some fonts, if necessary
\renewcommand{\namefont}{\Huge\rmfamily\bfseries}
\renewcommand{\personalinfofont}{\small\bfseries}
\renewcommand{\cvsectionfont}{\LARGE\rmfamily\bfseries}
\renewcommand{\cvsubsectionfont}{\large\bfseries}
\renewcommand{\quote}{\normalsize}

% Change the bullets for itemize and rating marker
% for \cvskill if you want to
\renewcommand{\itemmarker}{{\small\textbullet}}
\renewcommand{\ratingmarker}{\faCircle}

%% sample.bib contains your publications
%% \addbibresource{sample.bib}

%% [marc7s]: Multi language CV
% Change the language used in this file. 
% Use `\entrue` or `\svtrue` for English/Swedish
% Put in separate file for build script
\svtrue


\begin{document}
    \name{Martin Blom}
    \tagline{\en{Test Developer}\sv{CSE Student}}
    %% You can add multiple photos on the left or right
    %%\photoL{4cm}{john-doe}
    
    \personalinfo{
        \email{martinblom@live.se}\smallskip
        \phone{+46-737021018}
        \location{\en{City, Country}\sv{Göteborg, Sverige}}\\
        \linkedin{Martin Blom}{https://www.linkedin.com/in/martin-blom-b459a5231/}
        \github{Marre-B}
        %\homepage{nicolasomar.me}
        %\medium{nicolasomar}
        %% You MUST add the academicons option to \documentclass, then compile with LuaLaTeX or XeLaTeX, if you want to use \orcid or other academicons commands.
        % \orcid{0000-0000-0000-0000}
        %% You can add your own arbtrary detail with
        %% \printinfo{symbol}{detail}[optional hyperlink prefix]
        % \printinfo{\faPaw}{Hey ho!}[https://example.com/]
        %% Or you can declare your own field with
        %% \NewInfoFiled{fieldname}{symbol}[optional hyperlink prefix] and use it:
        % \NewInfoField{gitlab}{\faGitlab}[https://gitlab.com/]
        % \gitlab{your_id}
    }
    
    \makecvheader
    %% Depending on your tastes, you may want to make fonts of itemize environments slightly smaller
    % \AtBeginEnvironment{itemize}{\small}
    
    %% Set the left/right column width ratio
    \columnratio{0.26}

    % Start a 2-column paracol. Both the left and right columns will automatically
    % break across pages if things get too long.
    \begin{paracol}{2}
        % ----- STRENGTHS -----
        \cvsection{\en{Strengths}\sv{Styrkor}}
            \cvtag{Problemlösning} \\
            \cvtag{Stresstålig} \\
            \cvtag{Självsäker} \\
            \cvtag{Ansvarstagande} \\
            \cvtag{Noggrann} \\

            \medskip
            
        % ----- STRENGTHS -----
        
        % ----- CODING -----
        \cvsection{\en{Coding}\sv{Programspråk}}
            \cvtag{Haskell: Grundläggande} \\
            \cvtag{Java: Grundläggande} \\
            \cvtag{C: Grundläggande} \\
		\cvtag{C++: Grundläggande}

            \medskip
            
        % ----- CODING -----
        
        % ----- LANGUAGES -----
        \cvsection{\en{Languages}\sv{Språk}}
            \cvlang{\en{Lang 1}\sv{Svenska}}{\en{Native}\sv{Modersmål}}\\
            \medskip
            
            \cvlang{\en{Lang 2}\sv{Engelska}}{\en{Proficient / C2}\sv{Flytande}}
		\medskip
            
            \cvlang{\en{Lang 3}\sv{Kroatiska}}{\en{Proficient / C2}\sv{Grundläggande kunskaper}}
            %% Yeah I didn't spend too much time making all the
            %% spacing consistent... sorry. Use \smallskip, \medskip,
            %% \bigskip, \vpsace etc to make ajustments.
            \smallskip
        % ----- LANGUAGES -----
        
        % ----- KÖRKORT -----
        \cvsection{Körkort}
         \cvevent{\en{Title}\sv{Körkort}}{}{Oktober 2019}{}
            \en{
                \begin{itemize}
                    \item GPA: 1,23
                \end{itemize}
            }
            \sv{
                \begin{itemize}
                    \item B + AM Körkort
                \end{itemize}
	    }
 	    \divider
            
            \cvevent{\en{Title}\sv{Truckkort}}{| TLP10}{Februari 2022}{}
            \en{
                \begin{itemize}
                    \item GPA: 1,23
                \end{itemize}
            }
            \sv{
                \begin{itemize}
                    \item A1, A2, A3, A4, B1, B2, B3, B4
                \end{itemize}
	    }
        % ----- KÖRKORT -----

        % ----- REFERENCES -----
        \cvsection{\en{References}\sv{Referenser}}
            \sv{Andrea Leupold \\ Kundservice Chef \\ Volvo Car Retail \\
			\email{andrea.leupold@volvocarretail.se}
			\phone{08-580 00 633}
		}
            \smallskip
        % ----- REFERENCES -----
        
        % \cvsection{A Day of My Life}

	% ---  \cvachievement{\faTrophy}{Fantastic Achievement}{and some details about it}\\
        
        % Adapted from @Jake's answer from http://tex.stackexchange.com/a/82729/226
        % \wheelchart{outer radius}{inner radius}{
        % comma-separated list of value/text width/color/detail}
        % \wheelchart{1.5cm}{0.5cm}{%
        %   6/8em/accent!30/{Sleep,\\beautiful sleep},
        %   3/8em/accent!40/Hopeful novelist by night,
        %   8/8em/accent!60/Daytime job,
        %   2/10em/accent/Sports and relaxation,
        %   5/6em/accent!20/Spending time with family
        % }
        
        % use ONLY \newpage if you want to force a page break for
        % ONLY the current column
        \newpage
        
        %% Switch to the right column. This will now automatically move to the second
        %% page if the content is too long.
        \switchcolumn
        
        % ----- ABOUT ME -----
        \cvsection{\en{About Me}\sv{Om mig}}
            \begin{quote}
                \en{
                    Lorem ipsum dolor sit amet, consectetur adipiscing elit, sed do eiusmod tempor incididunt ut labore et dolore magna aliqua.
                }
                \sv{
                    Jag är en väldigt glad, utåtgående och allmänt kunnig person. Jag har ett stort intresse i datorer, programmering och övrig teknik,
			 jag är också intresserad av jakt och gillar fysiska utmaningar/röra på mig och vara i naturen. Jag pluggar för tillfället mitt andra år på Chalmers
			 Tekniska Högskola, med mål att ta en master's examen inom mjukvaruutveckling eller liknande.
                }
            \end{quote}
        % ----- ABOUT ME -----
        
        % ----- EXPERIENCE -----
        \cvsection{\en{Experience}\sv{Erfarenhet}}
            \cvevent{\en{Charge}\sv{Kioskansvarig}}{| \en{Company}\sv{Flen's Golfklubb}}{Juli 2018 -- Juli 2018}{\en{City, Country}\sv{Flen, Sverige}}
            \en{
                \begin{itemize}
                    \item First item
                    \item Second item
                    \item Third item
                \end{itemize}
            }
            \sv{
                	I kiosken så jobbade jag ensam, på eget initiativ och skötte alla uppgifter som att öppna upp på morgonen, skriva ned varor som behöver köpas in, 
			stå i kassan, vattna runt kiosken, se till att det alltid finns tillagade korvar och att stänga på kvällen.\\
            }
            \divider
            
            \cvevent{\en{Charge}\sv{Produktionschef}}{| \en{Company}\sv{McDonald's}}{September 2018 -- Augusti 2020}{\en{City, Country}\sv{Katrineholm, Sverige}}
            \en{
                \begin{itemize}
                    \item First item
                    \item Second item
                    \item Third item
                \end{itemize}
            }
            \sv{
                	På McDonald’s så jobbade jag till störst del i köket och gjorde då allt som tillkom, så som att se till att det fanns upptinade ingredienser, att alla maskiner var rena och fungerande och tillaga maten. 
			Jag är också utbildad inom alla serviceområden och stod ibland i kassan, plockade ihop beställningar och tog hand om drive-through’en. Jag har avklarad utbildning som produktionschef och 
			för att utbilda andra nya medarbetare, och hade en påbörjad utbildning för att kunna jobba som skiftledare. \\
            }
		\divider
		\cvevent{\en{Charge}\sv{Kundserviceagent}}{| \en{Company}\sv{Volvo Car Retail (Konsult - Manpower)}}{Maj 2022 -- XX}{\en{City, Country}\sv{Göteborg, Sverige}}
             \en{
                \begin{itemize}
                    \item First item
                    \item Second item
                    \item Third item
                \end{itemize}
            }
            \sv{
                	Här var jag anställd av Volvo Car Retail som konsult från Manpower. Jag jobbade på anläggningen Volvo Car Sörred i Torslanda området. I anläggningen hade vi både återförsäljning av nya och gamla bilar, samt en verkstad och biluthyrning. Min 
			arbetsuppgift som kundserviceagent involverade ett väldigt brett arbetsområde med mycket problemlösning och kundbemötande. Jag stod i "receptionen" till anläggningen och är första personen man får kontakt med när man kommer in. Därifrån är det 
			upp till mig att se till att kunden får 	den hjälp de behöver, eller kommer till rätt person. En vanlig dag kunde involvera saker så som, att beställa reservdelar, att boka verkstadstid åt en kund, att lämna ut/ta emot hyrbilar och arrangera provkörningar av 
			både nya och begagnade bilar. \\
            }
        % ----- EXPERIENCE -----
        
        % ----- EDUCATION -----
        \cvsection{\en{Education}\sv{Utbildning}}
            \cvevent{\en{Title}\sv{Teknikprogrammet}}{| Gymnasieskola}{Augusti 2017 -- Juni 2020}{\en{City, Country}\sv{Eskilstuna, Sverige}}
            \en{
                Attended \link{University}{https://university.com}
                \smallskip
                \begin{itemize}
                    \item GPA: 1,23
                \end{itemize}
            }
            \sv{
                Gick på \link{Grillska Gymnasiet}{https://grillska.se/grillska-gymnasiet-eskilstuna}
                \smallskip
                \begin{itemize}
                    \item Betygssnitt: 19.98
                \end{itemize}
            }
            \divider
            
            \cvevent{\en{Title}\sv{Data- och informationsteknik}}{| Universitet}{Augsti 2020 -- Juni 2025}{\en{City, Country}\sv{Göteborg, Sverige}}
            \en{
                \begin{itemize}
                    \item GPA: 1,23
                \end{itemize}
            }
            \sv{
		 Går på \link{Chalmers Tekniska Högskola}{https://www.chalmers.se/en/Pages/default.aspx}
		\smallskip
                \begin{itemize}
                    \item Betygssnitt: N/A
                \end{itemize}
            }
	    \divider

        % ----- EDUCATION -----
        
        % ----- PROJECTS -----
\begin{comment}
        \cvsection{\en{Projects}\sv{Projekt}}
            \cvevent{\en{Project 1}\sv{Projekt 1}}{\cvrepo{| \faGithub}{https://github.com/user/repo}\cvrepo{| \faGlobe}{https://repo-demo.com/}}{Mm YYYY -- Mm YYYY}{}
            \en{
                \begin{itemize}
                    \item Item 1
                    \item Item 2
                \end{itemize}
            }
            \sv{
                \begin{itemize}
                    \item Punkt 1
                    \item Punkt 2
                \end{itemize}
            }
            \divider
            
            \cvevent{\en{Project 2}\sv{Projekt 2}}{\cvrepo{| \faGitlab}{https://gitlab.com/user/repo}\cvrepo{| \faGlobe}{https://repo-demo.com/}}{Mm YYYY -- Mm YYYY}{}
            \en{
                \begin{itemize}
                    \item Item 1
                    \item Item 2
                \end{itemize}
            }
            \sv{
                \begin{itemize}
                    \item Punkt 1
                    \item Punkt 2
                \end{itemize}
            }
\end{comment}
            
        % ----- PROJECTS -----
    \end{paracol}
\end{document}
