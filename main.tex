%%%%%%%%%%%%%%%%%
% This is an sample CV template created using altacv.cls
% (v1.3, 10 May 2020) written by LianTze Lim (liantze@gmail.com). Now compiles with pdfLaTeX, XeLaTeX and LuaLaTeX.
% This fork/modified version has been made by Nicolás Omar González Passerino (nicolas.passerino@gmail.com, 15 Oct 2020)
%
%% It may be distributed and/or modified under the
%% conditions of the LaTeX Project Public License, either version 1.3
%% of this license or (at your option) any later version.
%% The latest version of this license is in
%%    http://www.latex-project.org/lppl.txt
%% and version 1.3 or later is part of all distributions of LaTeX
%% version 2003/12/01 or later.
%%%%%%%%%%%%%%%%

%% If you need to pass whatever options to xcolor
\PassOptionsToPackage{dvipsnames}{xcolor}

%% If you are using \orcid or academicons
%% icons, make sure you have the academicons
%% option here, and compile with XeLaTeX
%% or LuaLaTeX.
% \documentclass[10pt,a4paper,academicons]{altacv}

%% Use the "normalphoto" option if you want a normal photo instead of cropped to a circle
% \documentclass[10pt,a4paper,normalphoto]{altacv}

%% Fork: CV dark mode toggle enabler to use a inverted color palette.
%% Use the "darkmode" option if you want a color palette used to 
% \documentclass[10pt,a4paper,darkmode]{altacv}

\documentclass[10pt,a4paper,ragged2e,withhyper]{altacv}

%% AltaCV uses the fontawesome5 and academicons fonts
%% and packages.
%% See http://texdoc.net/pkg/fontawesome5 and http://texdoc.net/pkg/academicons for full list of symbols. You MUST compile with XeLaTeX or LuaLaTeX if you want to use academicons.

% Change the page layout if you need to
\geometry{left=1.2cm,right=1.2cm,top=1cm,bottom=1cm,columnsep=0.75cm}

% The paracol package lets you typeset columns of text in parallel
\usepackage{paracol}

% Change the font if you want to, depending on whether
% you're using pdflatex or xelatex/lualatex
\ifxetexorluatex
  % If using xelatex or lualatex:
  \setmainfont{Roboto Slab}
  \setsansfont{Lato}
  \renewcommand{\familydefault}{\sfdefault}
\else
  % If using pdflatex:
  \usepackage[rm]{roboto}
  \usepackage[defaultsans]{lato}
  % \usepackage{sourcesanspro}
  \renewcommand{\familydefault}{\sfdefault}
\fi

% Fork: Change the color codes to test your personal variant on any mode
\ifdarkmode%
  \definecolor{PrimaryColor}{HTML}{0F52D9}
  \definecolor{SecondaryColor}{HTML}{3F7FFF}
  \definecolor{ThirdColor}{HTML}{F3890B}
  \definecolor{BodyColor}{HTML}{ABABAB}
  \definecolor{EmphasisColor}{HTML}{ABA2A2}
  \definecolor{BackgroundColor}{HTML}{242424}
  \definecolor{AboutColor}{HTML}{ABABAB}
\else%
  \definecolor{PrimaryColor}{HTML}{001F5A}
  \definecolor{SecondaryColor}{HTML}{0039AC}
  \definecolor{ThirdColor}{HTML}{F3890B}
  \definecolor{BodyColor}{HTML}{666666}
  \definecolor{EmphasisColor}{HTML}{2E2E2E}
  \definecolor{BackgroundColor}{HTML}{E2E2E2}
  \definecolor{AboutColor}{HTML}{222222}
\fi%

\colorlet{name}{PrimaryColor}
\colorlet{tagline}{PrimaryColor}
\colorlet{about}{AboutColor}
\colorlet{heading}{PrimaryColor}
\colorlet{headingrule}{ThirdColor}
\colorlet{subheading}{SecondaryColor}
\colorlet{accent}{SecondaryColor}
\colorlet{emphasis}{EmphasisColor}
\colorlet{body}{BodyColor}
\colorlet{hyperlink}{PrimaryColor}
\pagecolor{BackgroundColor}

% Change some fonts, if necessary
\renewcommand{\namefont}{\Huge\rmfamily\bfseries}
\renewcommand{\personalinfofont}{\small\bfseries}
\renewcommand{\cvsectionfont}{\LARGE\rmfamily\bfseries}
\renewcommand{\cvsubsectionfont}{\large\bfseries}
\renewcommand{\quote}{\normalsize}

% Change the bullets for itemize and rating marker
% for \cvskill if you want to
\renewcommand{\itemmarker}{{\small\textbullet}}
\renewcommand{\ratingmarker}{\faCircle}

%% sample.bib contains your publications
%% \addbibresource{sample.bib}

% Multi language CV
% Change the language used in this file. 
% Use `\entrue` or `\svtrue` for English/Swedish
\entrue
% Put in separate file for build script
%\svtrue


\begin{document}
    \name{Martin Blom}
    \tagline{\en{Master's Student in High-Performance Computer Systems}\sv{CSE Student}}
    %% You can add multiple photos on the left or right
    %%\photoR{3cm}{john-doe}
    
    \personalinfo{
        \email{martinblom@live.se}\smallskip
        \phone{+46-737021018}
        \location{\en{Gothenburg, Sweden}\sv{Göteborg, Sverige}}\\
        \linkedin{Martin Blom}{martin-blom-b459a5231}
        \github{Marre-B}
        %\homepage{nicolasomar.me}
        %\medium{nicolasomar}
        %% You MUST add the academicons option to \documentclass, then compile with LuaLaTeX or XeLaTeX, if you want to use \orcid or other academicons commands.
        % \orcid{0000-0000-0000-0000}
        %% You can add your own arbtrary detail with
        %% \printinfo{symbol}{detail}[optional hyperlink prefix]
        % \printinfo{\faPaw}{Hey ho!}[https://example.com/]
        %% Or you can declare your own field with
        %% \NewInfoFiled{fieldname}{symbol}[optional hyperlink prefix] and use it:
        % \NewInfoField{gitlab}{\faGitlab}[https://gitlab.com/]
        % \gitlab{your_id}
    }
    
    \makecvheader
    %% Depending on your tastes, you may want to make fonts of itemize environments slightly smaller
    % \AtBeginEnvironment{itemize}{\small}
    
    %% Set the left/right column width ratio
    \columnratio{0.26}

    % Start a 2-column paracol. Both the left and right columns will automatically
    % break across pages if things get too long.
    \begin{paracol}{2}
        % ----- STRENGTHS -----
        \cvsection{\en{Strengths}\sv{Styrkor}}
	\sv{
            \cvtag{Problemlösning} \\
            \cvtag{Stresstålig} \\
            \cvtag{Självsäker} \\
            \cvtag{Ansvarstagande} \\
            \cvtag{Noggrann} \\
	}
	\en{
	Some specific abilities I have developed from work and experiences. \\
		\cvtag{Problem Solving} \\
		\cvtag{Good Communicator} \\
		\cvtag{Self Going} \\
		\cvtag{Used to Responsibilities} \\
		\cvtag{Leadership} \\
		\cvtag{Quick Learner} \\
	}

            \medskip
            
        % ----- STRENGTHS -----
        
        % ----- CODING -----
        \cvsection{\en{Coding}\sv{Programspråk}}
	\sv{
            \cvtag{Haskell: Grundläggande} \\
	 \cvtag{Python: Grundläggande} \\
            \cvtag{Java: Grundläggande} \\
            \cvtag{C: Grundläggande} \\
	 \cvtag{C++: Grundläggande} \\
	 \cvtag{C\#: Grundläggande}
	}
	\en{
       \cvtag{Java: Familiar} \\
	 \cvtag{Java/Type -Script: Familiar} \\
       \cvtag{C: Basic} \\
	\cvtag{Python: Experienced} \\
	 \cvtag{C++: Experienced} \\
	 \cvtag{C\#: Experienced} \\
	 \cvtag{MongoDB: Familiar} \\
	 \cvtag{SQLite: Experienced} \\
	 \cvtag{QT Widgets: Experienced} \\
	
	}

            \medskip
            
        % ----- CODING -----
        
        % ----- LANGUAGES -----
        \cvsection{\en{Languages}\sv{Språk}}
            \cvlang{\en{Swedish}\sv{Svenska}}{\en{Native}\sv{Modersmål}}\\
            \medskip
            
            \cvlang{\en{English}\sv{Engelska}}{\en{Proficient / Fluent}\sv{Flytande}}
		\medskip
            
            \cvlang{\en{Croatian}\sv{Kroatiska}}{\en{Basic knowledge}\sv{Grundläggande kunskaper}}
            %% Yeah I didn't spend too much time making all the
            %% spacing consistent... sorry. Use \smallskip, \medskip,
            %% \bigskip, \vpsace etc to make ajustments.
            \smallskip
        % ----- LANGUAGES -----
        
        % ----- KÖRKORT -----
        \sv{\cvsection{Körkort}}\en{\cvsection{Licenses}}
         \cvevent{\en{Driver License}\sv{Körkort}}{}{\sv{Oktober 2019}\en{October 2019}}{}

                \begin{itemize}
                    \item B + AM 
                \end{itemize}

 	    \divider
            
         \cvevent{\en{Forklift License}\sv{Truckkort}}{| TLP10}{\sv{Februari 2022}\en{February 2022}}{}

                \begin{itemize}
                    \item A1, A2, A3, A4, B1, B2, B3, B4
                \end{itemize}

        % ----- KÖRKORT -----

        % ----- REFERENCES -----
        \cvsection{\en{References}\sv{Referenser}}
            \sv{%Andrea Leupold \\ Kundservice Chef \\ Volvo Car Retail \\
			%\email{andrea.leupold@volvocarretail.se}
			%\phone{08-580 00 633}
		Referens kan lämnas på efterfrågan.
	}
	\en{
		References can be sent upon request.
	}
            \smallskip
        % ----- REFERENCES -----
        
        % \cvsection{A Day of My Life}

	% ---  \cvachievement{\faTrophy}{Fantastic Achievement}{and some details about it}\\
        
        % Adapted from @Jake's answer from http://tex.stackexchange.com/a/82729/226
        % \wheelchart{outer radius}{inner radius}{
        % comma-separated list of value/text width/color/detail}
        % \wheelchart{1.5cm}{0.5cm}{%
        %   6/8em/accent!30/{Sleep,\\beautiful sleep},
        %   3/8em/accent!40/Hopeful novelist by night,
        %   8/8em/accent!60/Daytime job,
        %   2/10em/accent/Sports and relaxation,
        %   5/6em/accent!20/Spending time with family
        % }
        
        % use ONLY \newpage if you want to force a page break for
        % ONLY the current column
        \newpage
        
        %% Switch to the right column. This will now automatically move to the second
        %% page if the content is too long.
        \switchcolumn
        
        % ----- ABOUT ME -----
        \cvsection{\en{About Me}\sv{Om mig}}
            \begin{quote}
                \en{
                    I'm a very happy, outgoing and generally knowledgeable person. I have a big interest in computers, programming and other tech ,but I also enjoy sports such as climbing and hunting.
			 At the moment I'm studying my second year of my masters in High Performance Computer Systems at Chalmers University of Technology, and looking for a future within the software industry.
                }
                \sv{
                    Jag är en väldigt glad, utåtgående och allmänt kunnig person. Jag har ett stort intresse i datorer, programmering och övrig teknik,
			 jag är också intresserad av jakt och gillar fysiska utmaningar/röra på mig och vara i naturen. Jag pluggar för tillfället mitt fjärde år på Chalmers
			 Tekniska Högskola där jag går mitt första år på masterprogrammet inom högpresterande datorsystem.
                }
            \end{quote}
        % ----- ABOUT ME -----

        % ----- EDUCATION -----
        \cvsection{\en{Education}\sv{Utbildning}}
            	\cvevent{\en{Computer Science}\sv{Data- och informationsteknik}}{\sv{| Universitet}\en{| University}}
			{\sv{Augsti 2020 -- Juni 2025}\en{August 2020 -- Present}}{\en{Gothenburg, Sweden}\sv{Göteborg, Sverige}}
            \en{
 	      Attending \link{Chalmers University of Technology}{https://www.chalmers.se/en/Pages/default.aspx}
             \smallskip
		\\
		Bachelor's Degree in Computer Science and Engineering \link{(August 2020 -- July 2023)}{}
		\\
		\smallskip
		Master's Degree in High-Performance Computer Systems \link{(August 2023 -- Present)}{}
		\\
		\smallskip
		Relevant Courses: Computer Architecture, Game Engine Architecture, Data Structures and Algorithms, Object-Oriented Programming, Machine-Oriented Programming, High-Perfomance Parallel Programming, Advanced Computer Graphics, Artificiell Intelligens
            }
            \sv{
	      Går på \link{Chalmers Tekniska Högskola}{https://www.chalmers.se/en/Pages/default.aspx}
	     \smallskip
                \begin{itemize}
                    \item Betygssnitt: N/A
                \end{itemize}
            }

        % ----- EDUCATION -----
        
        % ----- EXPERIENCE -----
        \cvsection{\en{Experience}\sv{Erfarenhet}}
            \cvevent{\en{McDonald's}\sv{Produktionschef}}{| \en{Production Manager}\sv{McDonald's}}
			{\sv{September 2018 -- Augusti 2020}\en{September 2018 -- August 2020}}{\en{Katrineholm, Sweden}\sv{Katrineholm, Sverige}}
            \en{}
            \sv{}

		\divider

		\cvevent{\en{Volvo Car Retail (Consultant - Manpower)}\sv{Kundserviceagent}}{| \en{Customer Service Agent}\sv{Volvo Car Retail (Konsult - Manpower)}}
			{\sv{Maj 2022 -- September 2022}\en{May 2022 -- September 2022}}{\en{Gothenburg, Sweden}\sv{Göteborg, Sverige}}
             \en{
                	At Volvo I was responsible of smoothly and with great creativity handling every day issues. These issues involved a very broad work area with a lot of problem solving and customer service. Usual task could include,
			 ordering extra parts, making a workshop reservation, receiving and giving out rental cars as well as arranging test drives of both new and used cars. \\
            }
            \sv{
                	Här var jag anställd av Volvo Car Retail som konsult från Manpower. Jag jobbade på anläggningen Volvo Car Sörred i Torslanda området. I anläggningen hade vi både
		 återförsäljning av nya och gamla bilar, samt en verkstad och biluthyrning. Min arbetsuppgift som kundserviceagent involverade ett väldigt brett arbetsområde med mycket
		 problemlösning och kundbemötande. Jag stod i "receptionen" till anläggningen och är första personen man får kontakt med när man kommer in. Därifrån är det upp till mig att 
		se till att kunden får den hjälp de behöver, eller kommer till rätt person. En vanlig dag kunde involvera saker så som, att beställa reservdelar, att boka verkstadstid åt en kund,
		 att lämna ut/ta emot hyrbilar och arrangera provkörningar av både nya och begagnade bilar. \\
            }

	 \divider

		\cvevent{\en{Wanda Sweden}\sv{Göteborg Representant}}{| \en{Gothenburg Representative}\sv{Wanda Sweden}}
			{\sv{Oktober 2022 -- Maj 2024 }\en{October 2022 --  May 2024}}{\en{Gothenburg, Sweden}\sv{Göteborg, Sverige}}
            \en{
               	Wanda opened their first warehouse in Gothenburg in October of 2022 and I was the first on-site employee. From there on I had the main responsibility for our operations in 
		Gothenburg. Aside from the regular work was I was tasked with vetting and educating new employees as well as making sure everything around the warehouse and our services ketp working. 
		Since August of 2023 I also had the responsibility of route planning/capacity management for all of the warehouses in Sweden. \\
            }
            \sv{
                	Wanda öppnade sitt första lager i Göteborg Oktober 2022 och jag belv den första annställda på plats. Jag har sedan dess haft mycket ansvar och varit huvudansvarig för vår
		 operation här i Göteborg då resterande av personalen -  inkluderat min närmsta chef -  i Sverige är belägna i Stockholm. Mitt ansvar utöver det vanliga arbetet har varit att lära upp
		 ny personal samt se till att allt kring lagret och våra tjänster fungerar och hålla min närmsta chef uppdaterad. Jag har sedan Augisti 2023 också varit ansvarig för rutt planering
		/kapacitet för alla våra lager i Sverige. \\
            }

	\divider

		\cvevent{\en{Volvo Car}\sv{Göteborg Representant}}{| \en{Software Developer}\sv{Wanda Sweden}}
			{\sv{Oktober 2022 -- Maj 2024 }\en{May 2024 --  August 2024}}{\en{Gothenburg, Sweden}\sv{Göteborg, Sverige}}
            \en{
               	 My work at Volvo Cars was as a summer intern supporting the Technical Leader of a team of around 15 people, in coding and processing data/models. During my time in this position I 
		single handedly wrote and developed a Python application for inhouse use as a database and data visualizer, to help improve the teams CFD workflow. The application was developed using mainly python libraries: SQLite and QTWidgets. \\
            }
            \sv{
                	
            }
            
        % ----- EXPERIENCE -----
        
        
    \end{paracol}
	\newpage
        % ----- PROJECTS -----

        \cvsection{\en{Projects}\sv{Projekt}}
            \cvevent{\en{Traffic Simulator}\sv{Projekt 1}}{\cvrepo{| \faGithub Visit Repo}{https://github.com/marc7s/TrafficSimulator}}{2023 -- 2024}{}
            \en{
                \begin{itemize}
                    \item Developed using C\# in Unity
                \end{itemize}
		
		This project was the main work of my Bachelor's thesis and aims to simulate traffic flow using an agent based system. The application features autonomous vechicles driving around a road network that can either be created by the user, or imported from a real world location.
            }
            \sv{
                \begin{itemize}
                    \item 
                    \item Punkt 2
                \end{itemize}
            }
		
            \divider
          
            \cvevent{\en{Game Engine}\sv{Projekt 2}}{\cvrepo{| \faGithub Visit Repo}{https://github.com/hannes44/LIDL-Engine}}{2024}{}
            \en{
                \begin{itemize}
                    \item Developed using C++ and OpenGL
                \end{itemize}
		\smallskip
		A 3D Game Engine with built in editor, written from scratch using OpenGL for graphics. Notable features:
		
		   \begin{itemize}
                    \item Physics Engine (Frame-rate independent)
			 \item Serialization
			 \item Scripting
			 \item Multiplayer
			 \item Graphical Interface Editor
                    \item Dynamic/Event based Input System
                \end{itemize}
            }
            \sv{
                \begin{itemize}
                    \item Punkt 1
                    \item Punkt 2
                \end{itemize}
            }
		\divider
          
            \cvevent{\en{Raytracing in DirectX12}\sv{Projekt 3}}{\cvrepo{| \faGithub Visit Repo}{https://github.com/v-olin/dx12-project}}{2024}{}
            \en{
                \begin{itemize}
                    \item Developed using C++/CUDA (DirectX12)
                \end{itemize}
		\smallskip
		A reatime raytracing rendering engine using Directx12. Notable features:
		
		   \begin{itemize}
                    \item Hardware accelerated raytracing
                    \item Ray traced ambient occlusion 
                    \item Transparent and reflective materials
                    \item Procedural terrain generation
                    \item Various basic compute shaders
                \end{itemize}
            }
            \sv{
                \begin{itemize}
                    \item Punkt 1
                    \item Punkt 2
                \end{itemize}
            }
		\divider
		These projects were completed in teams of 4 to 7 members, providing me with valuable experience in managing larger-scale initiatives and coordinating parallel or simultaneous development efforts. 
            
        % ----- PROJECTS -----

\end{document}
